% Options for packages loaded elsewhere
\PassOptionsToPackage{unicode}{hyperref}
\PassOptionsToPackage{hyphens}{url}
%
\documentclass[
]{article}
\usepackage{lmodern}
\usepackage{amssymb,amsmath}
\usepackage{ifxetex,ifluatex}
\ifnum 0\ifxetex 1\fi\ifluatex 1\fi=0 % if pdftex
  \usepackage[T1]{fontenc}
  \usepackage[utf8]{inputenc}
  \usepackage{textcomp} % provide euro and other symbols
\else % if luatex or xetex
  \usepackage{unicode-math}
  \defaultfontfeatures{Scale=MatchLowercase}
  \defaultfontfeatures[\rmfamily]{Ligatures=TeX,Scale=1}
\fi
% Use upquote if available, for straight quotes in verbatim environments
\IfFileExists{upquote.sty}{\usepackage{upquote}}{}
\IfFileExists{microtype.sty}{% use microtype if available
  \usepackage[]{microtype}
  \UseMicrotypeSet[protrusion]{basicmath} % disable protrusion for tt fonts
}{}
\makeatletter
\@ifundefined{KOMAClassName}{% if non-KOMA class
  \IfFileExists{parskip.sty}{%
    \usepackage{parskip}
  }{% else
    \setlength{\parindent}{0pt}
    \setlength{\parskip}{6pt plus 2pt minus 1pt}}
}{% if KOMA class
  \KOMAoptions{parskip=half}}
\makeatother
\usepackage{xcolor}
\IfFileExists{xurl.sty}{\usepackage{xurl}}{} % add URL line breaks if available
\IfFileExists{bookmark.sty}{\usepackage{bookmark}}{\usepackage{hyperref}}
\hypersetup{
  hidelinks,
  pdfcreator={LaTeX via pandoc}}
\urlstyle{same} % disable monospaced font for URLs
\usepackage[margin=1in]{geometry}
\usepackage{graphicx,grffile}
\makeatletter
\def\maxwidth{\ifdim\Gin@nat@width>\linewidth\linewidth\else\Gin@nat@width\fi}
\def\maxheight{\ifdim\Gin@nat@height>\textheight\textheight\else\Gin@nat@height\fi}
\makeatother
% Scale images if necessary, so that they will not overflow the page
% margins by default, and it is still possible to overwrite the defaults
% using explicit options in \includegraphics[width, height, ...]{}
\setkeys{Gin}{width=\maxwidth,height=\maxheight,keepaspectratio}
% Set default figure placement to htbp
\makeatletter
\def\fps@figure{htbp}
\makeatother
\setlength{\emergencystretch}{3em} % prevent overfull lines
\providecommand{\tightlist}{%
  \setlength{\itemsep}{0pt}\setlength{\parskip}{0pt}}
\setcounter{secnumdepth}{-\maxdimen} % remove section numbering

\author{}
\date{\vspace{-2.5em}}

\begin{document}

\hypertarget{how-many-people-are-there-in-each-enrollment}{%
\subsection{How many people are there in each
enrollment}\label{how-many-people-are-there-in-each-enrollment}}

\hypertarget{how-many-of-those-people-fail-the-first-question}{%
\subsection{How many of those people fail the first
question}\label{how-many-of-those-people-fail-the-first-question}}

\hypertarget{how-many-watched-the-video-about}{%
\subsection{How many watched the video
about}\label{how-many-watched-the-video-about}}

Welcome to the course Why would anyone want your data? Preserving
privacy in cloud storage: privacy by design Staying safe online:
personal perspectives Privacy online and offline

\hypertarget{how-many-of-those-got-the-quiz-correct}{%
\subsection{How many of those got the quiz
correct}\label{how-many-of-those-got-the-quiz-correct}}

\hypertarget{was-there-a-relation-between-if-you-watched-the-video-or-not-and-succeeding-the-quiz}{%
\subsection{Was there a relation between if you watched the video or not
and succeeding the
quiz}\label{was-there-a-relation-between-if-you-watched-the-video-or-not-and-succeeding-the-quiz}}

\hypertarget{total-views-by-each-year-for-each-video}{%
\subsubsection{Total views by each year for each
video}\label{total-views-by-each-year-for-each-video}}

\includegraphics{report--1-_files/figure-latex/unnamed-chunk-2-1.pdf}

\hypertarget{i-decided-to-see-if-there-was-a-relation-between-people-watching-the-videos-and-being-able-to-answer-the-questions-correctly.-therefore-i-took-the-data-from-the-enrollemnts.csv-to-check-for-each-year-the-amount-of-people-that-were-enrolled.-i-decided-to-check-for-the-last-4-years.}{%
\subsection{I decided to see if there was a relation between people
watching the videos and being able to answer the questions correctly.
Therefore i took the data from the enrollemnts.csv to check for each
year the amount of people that were enrolled. I decided to check for the
last 4
years.}\label{i-decided-to-see-if-there-was-a-relation-between-people-watching-the-videos-and-being-able-to-answer-the-questions-correctly.-therefore-i-took-the-data-from-the-enrollemnts.csv-to-check-for-each-year-the-amount-of-people-that-were-enrolled.-i-decided-to-check-for-the-last-4-years.}}

\hypertarget{year-7}{%
\subsubsection{Year 7}\label{year-7}}

\begin{verbatim}
## [1] 2342
\end{verbatim}

\hypertarget{year-6}{%
\subsubsection{Year 6}\label{year-6}}

\begin{verbatim}
## [1] 3175
\end{verbatim}

\hypertarget{year-5}{%
\subsubsection{Year 5}\label{year-5}}

\begin{verbatim}
## [1] 3544
\end{verbatim}

\hypertarget{year-4}{%
\subsubsection{Year 4}\label{year-4}}

\begin{verbatim}
## [1] 3992
\end{verbatim}

Then I found the amount of people that got the first question correct by
each year. I eliminated any any data that wasn't correctly inputted
(NA).

\hypertarget{year-7-1}{%
\subsubsection{Year 7}\label{year-7-1}}

\begin{verbatim}
## [1] 570
\end{verbatim}

\hypertarget{year-6-1}{%
\subsubsection{Year 6}\label{year-6-1}}

\begin{verbatim}
## [1] 619
\end{verbatim}

\hypertarget{year-5-1}{%
\subsubsection{Year 5}\label{year-5-1}}

\begin{verbatim}
## [1] 1040
\end{verbatim}

\hypertarget{year-4-1}{%
\subsubsection{Year 4}\label{year-4-1}}

\begin{verbatim}
## [1] 1034
\end{verbatim}

Then I decided to measure the views of total views for the 5 videos that
were uploaded and you were supposed to see before answering the quiz for
each year. I then found the average total views for all the 5 videos
combined.

\hypertarget{year-7-2}{%
\subsubsection{Year 7}\label{year-7-2}}

\begin{verbatim}
## [1] 629
\end{verbatim}

\hypertarget{year-6-2}{%
\subsubsection{Year 6}\label{year-6-2}}

\begin{verbatim}
## [1] 689.4
\end{verbatim}

\hypertarget{year-5-2}{%
\subsubsection{Year 5}\label{year-5-2}}

\begin{verbatim}
## [1] 1095.8
\end{verbatim}

\hypertarget{year-4-2}{%
\subsubsection{Year 4}\label{year-4-2}}

\begin{verbatim}
## [1] 1137.4
\end{verbatim}

I then decided to divide the mean mean videos over the enrollment number
so all the numbers will be proportionally equal.

\hypertarget{year-7-3}{%
\subsubsection{Year 7}\label{year-7-3}}

\begin{verbatim}
## [1] 0.2685739
\end{verbatim}

\hypertarget{year-6-3}{%
\subsubsection{Year 6}\label{year-6-3}}

\begin{verbatim}
## [1] 0.2171339
\end{verbatim}

\hypertarget{year-5-3}{%
\subsubsection{Year 5}\label{year-5-3}}

\begin{verbatim}
## [1] 0.3091986
\end{verbatim}

\hypertarget{year-4-3}{%
\subsubsection{Year 4}\label{year-4-3}}

\begin{verbatim}
## [1] 0.2849198
\end{verbatim}

From this data we can observe that most people that watched the people
proportionally wise was at the 5th year with the 6th year coming last by
a lot.

Now we are going to see who were the most people proportionally that got
the most correct answer. We are going to achieve this by diving the
question correctness by the enrollment size for each year.

\hypertarget{year-7-4}{%
\subsubsection{Year 7}\label{year-7-4}}

\begin{verbatim}
## [1] 0.2433817
\end{verbatim}

\hypertarget{year-6-4}{%
\subsubsection{Year 6}\label{year-6-4}}

\begin{verbatim}
## [1] 0.1949606
\end{verbatim}

\hypertarget{year-5-4}{%
\subsubsection{Year 5}\label{year-5-4}}

\begin{verbatim}
## [1] 0.2934537
\end{verbatim}

\hypertarget{year-4-4}{%
\subsubsection{Year 4}\label{year-4-4}}

\begin{verbatim}
## [1] 0.259018
\end{verbatim}

As we can see from the data that i have gathered we can see a very
similar result to the previous exercise that I did. This time again i
find the Year 5 to be the highest year with the most student that got
the result correct and the Year 6 to have the lowest correct results.
Furthermore the Year 4 and 7 were again very close in both aspects (
question/size and meanvideo/size) but also both of them are in the same
order as in the previous having year 4 coming second and year 7 coming
last.

These were 4 cases for this module. If it was possible i would increase
the data to be more precise with my measurements cause I know that 4
years are not that secure for the error handling but I can't cause the
data for the video stats doesn't exist before year 3th.

I will now plot a graph to show you the data more clearly.

\includegraphics{report--1-_files/figure-latex/unnamed-chunk-23-1.pdf}
\#\# Assumption

From this my assumption is that there is indeed a relation between the
time to give to watch a video about the subject and answering the quiz
question correctly even if the videos are not all relevant to the
question being asked. It seems that people that will do good at the quiz
are people that watch most if not all of the videos.

\hypertarget{what-type-of-device-did-they-use-and-did-it-have-any-relation-with-the-outcome}{%
\subsection{What type of device did they use and did it have any
relation with the
outcome?}\label{what-type-of-device-did-they-use-and-did-it-have-any-relation-with-the-outcome}}

\hypertarget{summary}{%
\section{Summary}\label{summary}}

I was given data for the video stats from the module cyber-security. I
extracted the statistics for device usage.

\hypertarget{assumptions}{%
\section{Assumptions}\label{assumptions}}

My Assumption is if there is a relation to what device you are using
with the likelihood that you are going to watch the videos.

\includegraphics{report--1-_files/figure-latex/unnamed-chunk-24-1.pdf}

\includegraphics{report--1-_files/figure-latex/unnamed-chunk-25-1.pdf}

\includegraphics{report--1-_files/figure-latex/unnamed-chunk-26-1.pdf}

We can observe that the majority of the people watch the videos from
their Desktop except for the 1st step in the video which is the
introduction of the course therefroe it doesn't have any important parts
they need to take into account. Year 6 was the highest for mobile and
the Year 5 was the highest for the tablet. Therefore lets compare how
they students did between 2.2 to 3 for tablet and mobile for all years.
cause the data is more constant around that region. Therefore the 3.11
quiz i believe is the most optimal one to choose with 3 multiple choices
cause we will see the where each student watch each video up to the
point of the quiz.

We find out through data wrangling that the number for correct answers
based on each year proportionally are the exact.

\hypertarget{year-7-5}{%
\subsubsection{Year 7}\label{year-7-5}}

\begin{verbatim}
## [1] 0.3142613
\end{verbatim}

\hypertarget{year-6-5}{%
\subsubsection{Year 6}\label{year-6-5}}

\begin{verbatim}
## [1] 0.2513386
\end{verbatim}

\hypertarget{year-5-5}{%
\subsubsection{Year 5}\label{year-5-5}}

\begin{verbatim}
## [1] 0.4616253
\end{verbatim}

\hypertarget{year-4-5}{%
\subsubsection{Year 4}\label{year-4-5}}

\begin{verbatim}
## [1] 0.4063126
\end{verbatim}

\includegraphics{report--1-_files/figure-latex/unnamed-chunk-31-1.pdf}
\includegraphics{report--1-_files/figure-latex/unnamed-chunk-32-1.pdf}
\includegraphics{report--1-_files/figure-latex/unnamed-chunk-33-1.pdf}

\#\#Summarise

There doesn't seem to be any relation between what device do you use and
how well your results are in the quiz.

\end{document}
